
\chapter{1. Problematika indoor map}



Orientační plány budov jsou pomůckou známou již několik staletí. Pomáhají lidem v orientaci v neznámém prostředí, typicky bývají schématické a zdůrazňují základní vybavení veřejné budovy – schodiště, chodby, vstupy či toalety. Nezřídka též obsahují označení místa, kde se člověk na plánu zrovna nachází – předzvěst modré geolokační ikonky v dnešních smartphonech.

Plánem je v kartografii nazýváno mapové zobrazení, které nerespektuje zakřivení země. V laickém významu ho ovšem používáme pro všechny detailní mapy malého území. V případě webových plánů budov se bude typicky jednat spíše o mapy, neboť využívají přímo zěmepisných souřadnic, tak jako objekty větších rozměrů.

Speciálním případem jsou plánky nouzového úniku z budovy, ten zvýrazňuje šipkami trasu k nejbližšímu únikovému východu. Pro nás opět pěkná předzvěst indoor navigace a routování.

S nástupem smartphonů v posledních 10 letech začal narůstat význam digitálních map, které kombinují navíc služby geolokace, geocodingu, routování a mnohdy i navigace. Samotná mapa pak často nabízí režim letecké mapy, šikmého snímkování, street view či tématické mapy pro cyklisty apod. Samozřejmostí jsou pak Bussiness Listingy s možností recenzování  Zjištění polohy uživatelů je pak cennou informací pro podnikání, zejména polohově cílenou reklamu, zjišťování návyků lidí či zcela nové možnosti v oblasti služeb. Získáváním výstupů z polohových dat se zabývá nově vzniklý vědní obor Location Intelligence, který samozřejmě využívá principy známé znalostnímu inženýrství (anglicky Bussiness Intelligence).

Indoor mapy jsou doménou několika málo posledních let. V roce 2012 přišel na scénu Google[1], o dva roky později Apple, načež mnohé servery komentovaly[2], že indoor bude “příští velká věc” (“the next big thing”).

Indoor mapy a geolokace je velká technologická výzva pro současný věk, v této kapitole si shrneme, jak k ní přistupují vybrané služby.



Obrázek: orientační plány budov ČVUT – FSv, FEL a FA[a]

1.1. Současné indoor služby



Současným indoor službám samozřejmě dominují velcí hráči na poli webových map. V posledních letech, též díky nástupu autonomních vozidel, proběhlo několik zajímavých akvizic. Pojďme si tedy shrnout – “kdo je kdo”:

TomTom – v roce 2008 akvizice firmy Teleatlas, nyní poskytuje data pro Apple Maps a Uber. Od roku 2014 navázal partnerství pro indoor mapy s firmou Micello.[3] 

Micello – vznikla v roce 2007 v Sillicon Valley a nyní má zmapováno největší množství indoor prostor. Data nabízí zejména do aplikací třetích stran.[4]

Google – do roku 2008 používal data TeleAtlasu, později využívá vlastní koupené či půjčené datasety[5]. Indoor mapy nyní implementuje do svých Google Maps na všech platformách.

Here – v roce 2011 vzniklo jako Nokia Ovi Maps akvizicí firmy NAVTEQ. V prosinci 2015 koupeno německými automobilkami od firmy Nokia už pod názvem Nokia Here Maps[6]. Firma NAVTEQ uveřejnila své indoor řešení na konferenci Nokia World 2010[7], ovšem Here Maps již tato data na webu nezobrazují.

Microsoft – služba Bing Maps oznámila indoor mapy v roce 2011 pod názvem Venue Maps[8]. Mapová data nadále vlastní, ale divize tvorby byla odprodána firmě Uber (po té co tato prohrála boj o Here Maps).[9]

Zcela vyčerpávající přehled lze najít v tabulce “The Indoor Navigation Market”[10] od společnosti BuildingLayer.

Google Maps Indoor



Google se stal konkurencí webových map akvizicí několika startupů v roce 2004, zejména firmy Keyhole, která dala za vznik legendární aplikaci Google Earth. Díky uvolnění API, mobilním aplikacím (včetně spolupráce s prvními telefony iPhone v letech 2007-2012) a Street-View se začal Google stávat jedním z hlavních hráčů na trhu. Celý poutavý příběh – viz zdroj.[11]



Obrázek: Zobrazení klasické mapy; přepnutí do indoor zobrazení; přiblížení (3x Google Maps)

Indoor mapám se začal věnovat od roku 2011, kdy spustil prohlížeč v rámci své aplikace pro Android.[12] Pro zadání mapových podkladů nabízí Google svou stránku, kde podporuje instituce, aby samy nahrály obrázky podlažních plánů a zarovnaly je vůči satelitní mapě. Samotné digitalizování pak dělá Google sám. Informační stránka – viz zdroj.[13] V ČR zatím nejsou žádné plány, ale instituce je i tak mohou Googlu poslat.[14]

Aktivace indoor zobrazení je nastavena jednak při dostatečném přiblížení mapy a též při kliku na Business Listing, který je umístěn v zmapované budově. V indoor zobrazení se též ukáže přepínač pater budovy.

Apple Maps Indoor



Společnost Apple v současné době neposkytuje indoor mapy. V průběhu roku 2015 proběhlo zarování umístění Business Listingů na jejich správné souřadnice v rámci budov nákupních center[15].

apple san fran.png

Obrázek: 3D zobrazení budovy; zarovnané ikony (2x Apple Maps)

Ovšem dle náznaků probíhají pilné přípravy. V listopadu 2015 zveřejnila aplikaci “Indoor Survey”[16], která umožňuje správcům velkých nákupních center provést zaměření pro vnitřní geolokaci. V aplikaci se v budově zaměří referenční body, které uloží otisk WiFi a Bluetooth vysílačů a data ze senzorů iPhonu.

Vice o speciálních geolokačních Bluetooth vysílačích Apple iBeacon v kapitole 1.2.

Není jisté, jestli Apple bude tvořit vlastní indoor mapy, nebo jen využívat přesnou indoor geolokaci pro aplikace třetích stran[17].

Microsoft Bing Maps





Obrázek: Zobrazení klasické mapy; přepnutí do indoor zobrazení; přiblížení (3x Bing Maps)

Společnost Microsoft spustila své webové mapy původně pod názvem Microsoft Virtual Earth na konci roku 2005. Ačkoliv v bohatství vlastností dostihuje Google Maps (letecké pohledy, street-view, 3D mapy, dopravní informace) zůstává stále několik kroků za ním. Chybí například podpora pro jiné mobilní platformy než Windows Phone. Pokrytí leteckých map či street-view je subjektivně horší.

Své indoor řešení pod názvem Bing Venue Maps spustila v roce 2011, zobrazení je podobné službě Micello, a tedy subjektivně přehlednější než Google. Obsahuje 5400 míst. Indoor režim se aktivuje kliknutím na zvýrazněnou budovu, přepínač pater je umístěn do drobečkové navigace v rámci mapy, dle mého mírně schovaný. Služba nabízí navíc Adresář Business Listingů v budově a též možnost procházet Venue Maps podle země a kategorie. Pro ČR tak eviduje 2 letiště, 45 nákupních center a pražskou zoologickou zahradu.

Služba vypadá v dnešní době mírně zastarale, Microsoft totiž posledních několik let vyvíjí tzv. Bing Maps Preview, které ovšem lze nainstalovat pouze jako Windows 10 aplikaci.

Micello



Společnost Micello se označuje jako vedoucího poskytovatele indoor map. Nabízí přes 25 000 zmapovaných míst,[18] v ČR je to asi desítka míst se subjektivně menší úrovní detailů než Bing. Společnost se specializuje zejména na prodej dat dále, vlastní aplikace nevyvíjí. Na jejím webu lze najít jen omezený prohlížeč.

Aktivace indoor se provádí kliknutím na ikonku budovy, chybí přímá integrace do okolní mapy. Ikonky jsou v tomto případě zobrazeny na API Google Maps. Na začátku roku 2016 oznámila spolupráci se startupem eeGeo, který by rád zmapoval celou zemi ve 3D.[19]

Jako jediný z komerečních projektů umožňuje omezenou editaci veřejností, uživatel může zadat až 3 poznámky (návrhy na přejmenování, rozdělení prostor apod.), které pak Micello projde a schválí. Dále jako jediný disponuje indoor routováním.

Ukázka dostupná na: www.micello.com/coverage 

micello moa 1.jpgmicello moa 2.jpg

Obrázek: 2x implementace v aplikaci “Mall of America”; 2x ukázka pokrytí na webu Micello[20]

Open-source platforma Anyplace



Otevřená platforma pod licencí MIT nabízí kompletní řešení pro indoor tvorbu, prohlížení i geolokaci.[21] Zdrojové kódy jsou dostupné na GitHubu pro všechny velké mobilní platformy, včetně nástroje pro zaměření zemského pole kompasem. Vývoj provádí skupina na University of Cyprus. Díky open-source povaze celého projektu se na něj můžeme podívat blíže, zejména s ohledem na možné využití některých komponent pro budoucí řešení nad platformou OpenStreetMap.

anyplace logger.pnganyplace viewer oc smíchov.png

anyplace viewer fa.pnganyplace webeditor.png

Obrázek: editace[22] a prohlížení FA ČVUT;  zobrazení plánu OC Smíchov; logger pro Android

Do Anyplace může po přihlášení kdokoliv nahrát podlažní plán své budovy, ten je třeba přesně zarovnat nad zobrazenou Google mapou. Následně je možné podlaží pojmenovat, přidat body zájmu a též vytvořit routovací graf (“Add new edge”). Zarovnání podlaží je však subjektivně velmi nepříjemné a není ani dobře možné zarovnat přidat vyšší podlaží na stejné místo.

Zobrazovací rozhraní Anyplace je implementováno také nad API Google Maps. Podobně jako v Micello jsou v klasické mapě ikonky zmapovaných budov, ovšem po rozkliku se budova v mapě jen elegantně překryje zvoleným plánem podlaží. Je možné prohledání všech bodů zájmu, jednoduché zobrazení trasy mezi body a možnost sdílet adresu či vložit mapku do jiného webu.

Celá aplikace včetně editace je postavena nad vlastním REST API, které je i dobře dokumentováno.[23] Nabízí se tak například i možnost teoretického importu-exportu dat mezi Anyplace a OpenStreetMap.

Mobilní aplikace pro platformy Android a Windows Phone jsou spíše proof-of-concept, ale nabídnou základní funkcionalitu. Android aplikaci lze přepnout též do režimu Logger, kdy zaznamenává data pro indoor geolokaci na základě Wi-Fi, kompasu a IMU (viz. kapitola 1.2).

Anyplace se tedy jeví jako velmi vhodná platforma pro případnou užší spolupráci, zejména v podobě využítí některých služeb pro budoucí napojení na OpenStreetMap.

Více informací na github.com/dmsl/anyplace 

Codrops Interactive 3D Mall Map



Tento koncept moderního prohlížeče indoor plánu nákupního centra je vytvořen pomocí technologií CSS a SVG. Projekt je k dispozici jako open-source na GitHubu. Licence podobná CC-BY-SA[b][c][d][e].

Sice nejde o mapovou aplikaci, ale i tak nabízí zajímavou inspiraci z hlediska UX. Možné využíti je diskutováno v kapitole 4[f].

Více informací na github.com/codrops/Interactive3DMallMap



Obrázek: Vizualizace pater nákupního centra (2x Codrops)

ČVUT Navigátor[g]



Před několika lety na ČVUT již vznikla platforma pro správu a zobrazení indoor map zejména pro navigaci,[24] nyní již není v provozu. Výslednou mobilní aplikaci pěkně prezentuje video na YouTube[25] – zahrnovala rozvrh studenta, klasickou mětskou mapu pro vyhledání budovy, navigační indoor mapy a geolokaci pomocí QR kódů.

Plány jednotlivých podlaží byly rastrové, nad nimiž byl ručně vytvořen routovací graf. Po spuštění navigace bylo možné naskenovat QR kód, který se na obrazovce změnil na navigační šipku. Datový zdroj byla serverová aplikace nad speciálně navrhutým formátem a umožňovala základní možnosti úprav grafu a nahrávání rastrových plánů.

Tento rok[h] bylo též zadáno několik závěrečných prací pod názvem ČVUT Navigátor 3.0. Soustředí se na různé klientské aplikace pro indoor navigaci. Bylo vyspecifikováno nové vektorové API i datový zdroj nad CMS ModX.

Naše práce řeší obecný problém získávání a správy indoor map v rámci otevřené platformy OpenStreetMap, tedy může být v budoucnu zdrojem pro libovolné navigační řešení. [i]

i-locate consorcium a IndoorGML formát



V souvislosti s touto prací je třeba zmínit program podpořený evropskou unií, který si klade za cíl vytvořit “protějšek OpenStreetMap pro indoor použití”.[26] Projekt trvá od začátku roku 2014 do konce 2016 a měl by vytvořit infrastrukturu, nástroje a formáty pro open-data i crowd-sourcing.

Důležitou částí projektu je IndoorGML formát – standard přijatý OpenGeo Consorciem.[27] Měl by sloužit jako otevřený XML formát pro výměnu a modelování indoor dat, včetně možnosti pro využití routingu a napojení na na struktury vnějšího světa, tedy například formát CityGML. Editor IndoorGML je postaven jako plugin[28] editoru JOSM (více v kapitole 2.2.).

Google Project Tango[j]



Google se již delší dobu zabývá přesným modelováním prostoru, trasování polohy a vnímání hloubky.[29] Užití je vhodné zejména pro dokonalé augmented reality aplikace, ale počítá se i s crowd-sourcingem vnitřních prostorů pro mapy. V únoru 2016 nabídli působivé demo[30], kde tablet při procházení muzeem živě modeluje okolní 3D prostor, zná svojí polohu s přesností na centimetry a díky tomu ukazuje v augmented reality na zemi navigační prvky. Tento systém sice vyžaduje speciální podporu hardwaru (přesnější senzory, kamery, výkon atd), ale velmi dobře ilustruje, kam by se mohla technologie indoor geolokace a navigace ubírat.



Obrázek: Ilustrační obrázek rozpoznáného 3D profilu a zaznamenané stopy

1.2. Geolokace



Pro venkovní použití se nejčastěji mluví o satelitní navigaci – tedy dopočítání polohy triangulací z obdržených časových razítek několika satelitů. Pro určení polohy (tzv. fix) je třeba přímá viditelnost alespoň tří po 12,5 minuty[31] při studeném startu. Dnešní zařízení často implementují několik paralelních systémů (zejména americký Navstar GPS, ruský Glonass, nebo evropský Galileo) a systém A-GPS – tedy předstažení obecných dat pomocí mobilního internetu. Tím je dosaženo vyšší přesnosti a též možnost určení polohy i v zarušeném prostředí. Uvnitř budovy je tento systém téměř nepoužitelný, neboť chybí přímá viditelnost na oblohu. Může být dosaženo pouze náhodného fixu, často s velkou nepřesností.

Další venkovní systém je zaměření GSM vysílačů (tj. BTS), ten s jistou úspěšností funguje i v budovách, ale krom bezpečnostních složek se v praxi nepoužívá, neboť operátoři dovolují připojení pouze k jednomu BTS.[k][l][m][n] Poloha je pak kruh kolem připojené BTS s poloměrem stovky metrů až desítek kilometrů.

Posledním venkovním systémem je triangulace Wi-Fi vysílačů[32], využívá hustého pokrytí tímto signálem v zastavěných městských oblastech. Ovšem data musí být proměřena a shromážděna. Kvůli aktualizaci i případné telemetrii (měření pomocí připojených klientských zařízení) jsou tato data typicky uložena v on-line centralizovaném úložišti[o][p][q][r][s]. Tento systém zaznamenal velký rozmach s nástupem smartphonů, zejména díky implementaci v systémech Android v podobě “Google Location Services”. Tam musí uživatel souhlasit, že jeho data budou použita ke zpřesňování služeb, tedy telemetrii. Tento systém se i velmi výhodně využívá uvnitř budov.

Pro indoor geolokaci (též Indoor Position System – IPS) navíc vzniklo v poslední době několik dalších systémů a rozšíření:

triangulace WiFi doplněná o frekvence Bluetooth – typicky se jedná o standard BT LE 4.0, který umožňuje vytvořit malé vysílače s provozem na baterii několik let. Kromě existující infrastruktury Wi-Fi je ovšem nutné pořídit tyto malé vysílače a rozmístit je po budově. Což pro veřejné instituce a rozlehlé vícepatrové budovy může být finančně náročné. 



Kromě čistě identifikačních beaconů se často připojují i další místně relevatntní informace, např.: URL. Pod „buzzwordem“ Physical Web tyto funkce nabízí Google Eddystone (jako open-source), Apple iBeacon (pouze pro zařízení této značky) či další produkty kombinující i více protokolů.

měření anomálii magnetické charakteristiky zemského pole – to je snadno dostupné přes běžný digitální kompas. Tato technologie nevyžaduje finanční investici, ovšem je třeba projít všechna místa v budově a přitom volit na mapě referenční body. Opět vyžaduje uložení dat typicky v cloudu[t] a též obnovovací měření při změně kovových struktur budovy.[33]



Opensource řešení nabízí Anyplace, komerční například indooratlas.com.

“dead reckoning”[34] – jedná se o odhad geolokace na základě poslední určené polohy a směru pohybu. Ve smartphonech lze využít akcelerometru (IMU), ovšem nepřesnost vinou sčítání chyby neustále narůstá.

Další méně používané metody[35] zahrnují: hlukový otisk místnosti[u][v][w][x][y][36], čtečku QR kódů[z][aa][ab][ac], krokměr či dvoukolový odometr pro vozidla či vozíky.

Současné toolkity Google Location Services a Apple Core Location kombinují více uvedených technik jak pro venkovní použití tak pro indoor.

1.3. (Tvorba podkladů)



// aneb pokud zbude čas a bude to chtít větší rozsah.

Google Cartographer batoh



// http://techcrunch.com/2014/09/04/google-unveils-the-cartographer-its-indoor-mapping-backpack/

Komunitní tvorba - position mouse



// http://atterer.org/leadme (Naivní) Nástroj pro měření vzdáleností pro tvorbu indoor mapy

 

(Sensopia Magicplan)



// aplikace na zaměřování prostor pomocí kamery

2. Řešení nad platformou OpenStreetMap[ad]



Otevřená mapa světa – OpenStreetMap – je projekt, který si klade za cíl vytvořit crowd-sourcovanou mapu světa pod svobodnou licencí. V mnohém je tedy podobný dobře známé encyklopedii Wikipedia. Vznikl v roce 2004 v Anglii, zejména z důvodu nedostupnosti vhodných dat pro satelitní navigace. O čtyři roky později byl projekt zastřešen pod neziskovou organizaci OpenStreetMap Foundation. Postupně do projektu data přibývala, velký úspěch zaznamenal v roce 2012, kdy ho začalo používat několik velikých společností z důvodu zpoplatnění Google Maps API. Od té doby roste úspěšně dál.



Obrázek: Počet registrovaných uživatelů (log. měřítko); aktivní uživatelé za měsíc (modrá)[37]

Technicky se jedná o vektorovou mapovou databázi, kterou může upravovat každý zaregistrovaný uživatel. Pokud se tedy někdo rozhodne zmapovat třeba všechna piána na ulicích, může si na to vytvořit vlastní značku a do databáze je přidat. Omezení je jediné, aby objekt byl fyzicky přítomný v reálném světě. Což samozřejmě platí i pro vnitřky budov.

Ovšem aby to vše dobře fungovalo, udržuje světová komunita OpenStreetMap společný značkovací klíč, který určuje, jak se budou určité objekty značit globálně. Díky tomu je možné vykreslit silniční síť celého světa, či najít cyklistickou trasu z Madridu na Sibiř. Ke společné shodě dochází pomocí procesu tzv. proposals.

Značkovací klíč – též tagování – je k dispozici v českém překladu na wiki.osm.org/Cs:Map\_Features. Obsahuje hlavní kategorie všech objektů, další upřesnění jsou k dispozici na stránkách jednotlivých značek. Ovšem to, že je například lavička ve značkovém klíči, neznamená, že by v OpenStreetMap byly zaneseny všechny lavičky na světě. To vždy záleží na místní komunitě a zejména konkrétních přispěvatelích, co se kdo rozhodne mapovat. Česká komunita je jedna z těch aktivnějších, a tak jsou v ČR zmapovány všechny hlavní silnice, čísla popisná či třeba velké množství turistických značek a cyklotras.[38]

Indoor mapování je pro OpenStreetMap přirozená výzva, a proto nepřekvapí, že nemalé úsilí již bylo tímto směrem vyvinuto. Do nynějška ovšem žádný proposal přijatý nebyl, proto se v této práci pokusíme využít získaných poznatků, poučit se z chyb a navrhnout řešení, které bude mít šanci na úspěch. Více v kapitole 2.4.

Tak jako Steve Coast měl v roce 2004 vizi s celosvětovou mapou ulic, tak my nyní můžeme předestřít vizi celosvětové indoor mapy tvořené uživateli. Výhody jsou nasnadě, z volných geodat a jednotného formátu pro celý svět může vzniknout množství velmi zajímavých aplikací. Příklady jsou k vidění už dnes. Za všechny uveďme aplikaci Maps.me – pro všechny mobilní platformy umožňuje offline zobrazení mapy včetně 3D budov, navigaci či editaci bodů zájmu uživateli.

Otevřená data vždy přilákají mnoho firem a tvůrců aplikací, a díky tomu vzniknou i nové a nečekané využití. Může se jednat o různé prohlížeče, aplikace veřejných institucí pro návštěvníky s integrací mapy, prodejce nemovitostí, navigaci osob se sníženou schopností orientace, či zcela globální prohlížeč map včetně indoor režimu.

2.1. Core architektura



Pojďme si nyní ukázat, na čem je OpenStreetMap postaveno – vnitřní způsob ukládání dat, API a základní nástroje pro pokročilou práci.

Mapová data nejsou jako v klasickém GISu objekty s geometrií, ale využívá se specifické schéma, které má tu výhodu, že zachovává topologii a je tak přímo vhodné jako routovací graf. Základní prvky jsou tyto:

uzel (node) – bod, který jako jediný nese souřadnice

cesta (way) – posloupnost referencí na uzly. POZOR, nejde o cestu v běžné řeči, ale v grafové terminologii. Význam jí dodávají až aplikované tagy.  

relace (relation) – reference více uzlů, cest a relací

Samotný význam je pak možné ke každému primitivnímu prvku přidávat v podobě libovolných značek – tagů. Krom toho ještě cesta může znamenat i plochu – to v případě že je uzavřená a nese určité tagy (třeba area=yes). Tímto je tedy výčet úplný:

tag (tag) – klíč a hodnota (např. highway=residential)        

plocha (area) – uzavřená cesta, která nese určité tagy

metadata – doplňkové informace pro správu dat (autor, id, verze, changeset)

Editace se provádí pomocí API tak, že uživatel pošle upravené prvky v rámci jednoho změnového požadavku – tzv. changesetu. V databázi se pak vytvoří nové prvky a upraví ty stávající zvýšením verze. Smazání se provede nastavením flagu visible a technicky tedy jde o úpravu. V případě vandalské editace je tak možné celý changeset odvolat (revert) a vrátit data do původní polohy.

Data a OSM XML



Data se ve většině případů reprezentují ve formátu OSM XML, uveďme si jednoduchý příklad. Obsahuje tři uzly, z nichž rozcestník stojí zcela samostatně a dva uzly jsou referencovány v rámci cesty. Jeden z těchto uzlů v cestě nese zároveň další vlastnost – pomníček.

Relace sdružuje všechny zmíněné prvky a vytváří turistickou trasu “Odbočka k pomíčku”, kde definuje její barvu, referenční číslo i symbol, který má být vykreslen.

<osm version='0.6' upload='true' generator='JOSM'>

 <node id='370617054' … lat='50.0995937' lon='14.3048577'>

   <tag k='tourism' v='information' />

   <tag k='information' v='guidepost' />

   <tag k='hiking' v='yes' />

   <tag k='name' v='Vaníčkův pomníček' />

   <tag k='operator' v='cz:KČT' />

   <tag k='ref' v='6014/7' />

 </node>

  <node id='370559023' … lat='50.0995577' lon='14.3048705' />



 <node id='370559037' … lat='50.0992923' lon='14.3049849'>

    <tag k='historic' v='memorial' />

    <tag k='memorial' v='stone' />

   <tag k='inscription' v='Zde zemřel sokolský písmák Karel Vaníček 24. června r. 1926' />

   <tag k='name' v='Vaníčkův pomníček' />

 </node>



 <way id='32902919' … >

   <nd ref='370559023' />

   <nd ref='370559037' />

   <tag k='highway' v='path' />

 </way>



 <relation id='6065569' timestamp='…' uid='162287' user='zby-cz' visible='true' version='3' changeset='39139040'>

   <member type='node' ref='370617054' role='guidepost' />

   <member type='way' ref='32902919' role='' />

   <tag k='complete' v='yes' />

   <tag k='description' v='Odbočka k Vaníčkovu pomníčku' />

   <tag k='distance' v='0.05' />

   <tag k='kct\_yellow' v='interesting\_object' />

   <tag k='network' v='lwn' />

   <tag k='operator' v='cz:KČT' />

   <tag k='osmc:symbol' v='yellow:white:yellow\_turned\_T' />

   <tag k='ref' v='6014i' />

   <tag k='route' v='hiking' />

   <tag k='type' v='route' />

 </relation>

</osm>

Obrázek: Ukázka formátu OSM XML[39] a reprezentace v editoru JOSM

Informace o topologii – napojení cest – je tak dána referencováním ID stejného uzlu. Již je ponécháno na konzumentovi dat, aby získal pouze relevantní prvky (typicky cesty s tagem highway=*) a následně si z dotažených uzlů vynesl geometrii cest a spočítal ohodnocení hran.

Relace v principu přenáší své tagy na jednotlivé členy. Doporučuje se používat pouze tam, kde je skutečně potřeba. Kvůli horší možnosti editace je například nevhodná pro sdružovat více cest tvořící jednu ulici. Naopak rozsáhlejší sdružení objektů, typicky turistické trasy, jsou v relacích vítané, neboť se lépe udržují a též je umožněno využít jedné fyzické cesty pro vedení více tras. Každý člen může mít uvedenou roli vůči relaci.

Nejčastěji používané typy jsou tyto, určují se tagem type=*:

route – viz příklad – vedení turistických, cyklistických tras, či trasy MHD, trajektů apod.

multipolygon – jednak umožňuje ploše vymezit, kde se nenachází (tj. ohraničit díru) a jednak pro rozsáhlé plochy (lesní porost) umožňuje sdružit více okrajových cest. Ty jsou děleny z důvodu snazší práce v editorech.

boundary – sdružuje okrajové cesty územních hranic a též relace podřízených oblastí. Například relace České republiky bude obsahovat všechny hraniční cesty[ae] (ways) a relace všech krajů.

restriction – zaznamenává zákazy odbočení, které implicitně neplynou z jednosměrek.

Díky velké rozmanitosti získávání dat je možné narazit na různé zápisy téhož. Zejména bod zájmu – třeba pomníček – může být ve vztahu k přísupové cestě[af][ag][ah] reprezentován:

uzlem (na konci pěšiny i mimo ní) – přibližná poloha objektu

plochou (napojenou i nenapojenou na cestu) – značící přesnější obrys fyzického objektu

relací multipolygon (opět napojenou nebo jen “poblíž”) – značící složitější plochu

API a další služby



Samotná databáze OpenStreetMap je hlavní centralizovaný zdroj, kam se ukládají všechny editace. Aby bylo možné mapová data dále používat, existuje několik služeb, která je poskytují navenek.

API – obsahuje nejčerstvější data a je určeno pouze pro účely editace. Umí vrátit všechna data v požadovaném výřezu (viz[ai] [40]) a uložit upravené objekty zpět. Krom toho nabízí endpointy pro práci s historií, changesety a uživatelskými profily. Vše též po OAuth1.0 či jiné autentifikaci[aj]. Komunikuje pomocí OSM XML. Více na wiki.osm.org/API

Planet dump – jednou týdně proběhne export celé databáze ve formátu OSM XML s kompresí bzip2 a Google protobuf formátu (osm.pbf). K dispozici jsou jen “aktuální verze” (49 GB, resp. 32 GB), vše včetně historie (75 GB, resp. 51 GB). Též se nabízí různé změnové soubory (osc) v týdením, denním i minutovém rozsahu. Více na planet.openstreetmap.org

Overpass API – rozhraní optimalizované na čtení data a vykonávání i poměrně složitých dotazů. Funguje v zásadě jako databáze přes protokol http. Hodí se pro jednorázové exporty zájmových skupin prvků i pro pravidelné využití v podobě backendu různých webových aplikací. Umožňuje dva dotazovací jazyky a výstup v OSM XML či JSON, bohužel chybí přímá podpora pro GeoJSON. 

Více informací na wiki.osm.org/Overpass\_API, grafické rozhraní na overpass-turbo.eu

XAPI – kopie API pouze pro čtení s povolenou větší zátěží. Dnes nahrazena Overpassem.

Exporty – několik dalších serverů nabízí předzpracovaná data z exportu planety. Metro etracts nabízí vyříznuté oblasti velkých měst[41], Geofabrik.de nabízí výřezy pro jednotlivé kontinenty a státy[42].

Databáze



Hlavní web openstreetmap.org, který obsahuje i API, je napsán ve frameworku Ruby on Rails a přímo komunikuje s databází, podívejme se na ní blíže.

Jedná se o databázi PostgreSQL s rozšířením PostGIS pro pokročilou práci s prostorovými daty.

Pod postgis

osmosis

core: Databázi postgis ve WGS84, polohopisná

Relace(!), data, tagování, osm-xml, api (overpass, xapi, planet,..)

Renderery



xyz, rastr tiles, nejednost ikonek, problémy s velikostí, mapnik, mercator, vrstevnice z nasy

2.2. Editory



Tak jako existuje mnoho služeb, vzniklo komunitním vývojem i přes desítku editorů pro úpravu OpenStreetMap. Podívejme se pouze na ty současné, na jejich charakteristiky i případné schopnosti tvořit indoor mapy.

JOSM



JOSM je pokročilý editor, určený pro power-usery. Je postaven na platformě Java, takže ho lze spustit na všech operačních systémech. Vývoj začal v roce 2006 a od začátku se vyznačoval velikou rozšiřitelností. V současnosti čítá přes 200 pluginů, 100 rozšířených předvoleb tagování (presets) a mnoho vykreslovacích stylů. Velikou nevýhodou je nepřehlednost uživatelského rozhraní, proto není vhodný pro začátečníky.

Využívá společný editor-layer-index[43] pro definici mapových a ortofoto vrstev, ze kterých je možné odvozovat data. Tam se kromě serverů s globálním pokrytím nachází i mnoho regionálních, které jsou nabídnuty podle aktuálního místa editace.

Vyznačuje se možností práce offline a tedy i načítání větších datasetů, či mnoha záznamu GPS trasy (GPX soubory). Kromě načtení klasického výřezu z API umí i stahovat jednolivé prvky, či kompletní relace. Též obsahuje přímé napojení na Overpass API. Pomocí pluginů pak například načítá fotky z otevřeného street-view Mapillary, či vykresluje výšky objektů dle metodiky Simple 3D.

Pro pohodlnější editaci nabízí možnost vlastního filtrování dat dle zadaného dotazu. Tato vlastnost je základní funkcionalita nutná pro indoor mapování. Tagovací předvolby a vykreslovací styly pro některé indoor metodiky jsou též již k dispozici, případně je možné je snadno vytvořit.



Obrázek: Editor JOSM

Potlatch2



Online webový editor na platformě Flash, který byl v aktivním vývoji 2009-2013. Na vlastním vykreslovacím jádře umí svižně zobrazovat i editovat data a disponuje i několika pokročilými funkcemi. Nyní už je na ústupu, neboť neumožňuje tak snadnou rozšiřitelnost a též protože technologie Flash přestává být podporována ve prospěch HTML5.

Pro indoor mapování nemá podporu – vykresluje prvky všech pater současně.



Obrázek: Editor Potlatch2 s rozbalenou nabídkou podkladů



Editor iD



Nový webový editor iD je postaven na technologiích HTML5. Úsilí začalo v roce 2012, kdy byl podpořen grantem Knights foundation[44] a vývoj koordinován společností MapBox. Hlavní vývoj proběhl v roce 2013, kdy byl i nasazen jako výchozí editor na stránce openstreetmap.org.

iD je narozdíl od ostatních editorů navržen zejména s ohledem na jednoduchost použití. Po spuštění nabídne uživateli interaktivní komentovanou prohlídku základů editace. Jasně odlišuje kreslící a prohlížecí módy a nabízí bohaté formuláře s přednastaveným tagování (presets). V režimu přímého zadávání tagů pak nabízí již existující tagy dle databáze taginfo.



Obrázek: Editor iD s vybraným uzlem rozcesníku

Z technologií HTML5 využívá pokročilou vizualizační knihovnu d3.js. Vykreslení přenechává na inline SVG, které lze výhodně stylovat pomocí klasického CSS3. Díky tomu lze editoru snadno upravovat chování za běhu a vyvíjet nové vlastnosti (tzv. hackable editor).

Pro indoor mapování nemá podporu – vykresluje všechny prvky všech pater současně.

2.3. (OSM as a platform)



https://speakerdeck.com/olafveerman/osm-as-a-platform

Instalace OSM-website:

věc postavená na ruby-on-rails

instalace závislostí na osx trochu těžší

bundle install

gem update jwt vXXX - protože referencovaná verze byla “yanked”

rozbitý therubyracer (pro node) - kompilace na macu

potom už snadné dle návodu 

založení DB (postgre (postgis?)

osmosis --read-xml ~/Desktop/aachen.osm --write-apidb host="localhost" database="openstreetmap" user="openstreetmap" password="openstreetmap" validateSchemaVersion="no"

2.4. Indoor



Jak už bylo naznačeno v úvodu této kapitoly, indoor mapování v OpenStreetMap již prošlo určitým vývojem. Vzniklo několik proposals, které se vyzkoušely v praxi, proběhlo několik pokusů o tvorbu editoru a neposledně bylo zmapováno stovky budov po celém světě (ačkoliv, s trochou nadsázky, každá mírně jinou metodikou).

Indoor mapování je specifické zejména potřebou oddělovat jednotlivá patra a seskupovat další prvky podle nich.[ak]

Tato práce se nesnaží o nalezení zcela nového způsobu tagování, ale staví na tom co je dobře navrženo a přidává i ubírá tak, aby výsledek byl přijatelný pro komunitu.

Výzvy pro indoor mapování



Architektura OpenStreetMap je navržená velmi flexibilně a nemá potíž se začleněním indoor dat, ovšem jsou zde některé technické i netechnické výzvy, které je dobré promyslet. Na OSM wiki[45] je sepsal XXXX[al].

Data jsou svým charakterem trochu jiná – nekreslí se v jedné vrstvě, ale každé patro má svou vrstvu. Naopak klasické 2D mapy jsou nyní ploché, zobrazují vše v jedné vrstvě. V případě bodů zájmu to může být žádoucí, pokud jich není mnoho na stejných souřadnicích, ovšem v případě překrytí mírně jiných cest[am] vznikne nesrozumitelný výstup. Způsobeno je to také tím, že pro některé indoor i venkovní prvky (například chodník) se používá stejné tagování, takže renderer je vykreslí v obou případech. 

Editory vždy vykreslují všechna data, takže problém je ještě markantnější.

Bussiness listingy se, obzvlášť v nákupních centrech, mohou často měnit, geometrie však zůstává. Tento bod adresuje obecnější problém, jak se má OSM postavit ke katalogizaci firem. V ideálním případě by existoval volný katalog firem se všemi údaji, otvírací dobou apod. a v OSM by byla uložená jen geometrie + prolinkování do katalogu. Ovšem naděje na vznik podobného katalogu jsou mizivé.

Právní důsledky – mapování veřejného prostoru je v pořádku, ovšem například nákupní centra bývají soukromě vlastněné objekty, do kterých je jen veřejnosti umožněn vstup. Mělo by se na to pamatovat.

Dočasné mapy – například pro různá výstaviště je typické, že každoročně rozestavuje stejné schéma pro jeden konkrétní veletrh. V tuto chvíli není vhodný způsob jak tuto informaci do OSM uložit. Pokud se geometrie v databázi zanechají s jinými tagy a přes ně se budou kreslit data pro “aktuální veletrh”, bude v databázi nepřehledná změť čar. Toto je opět výzva pro editory i mapové výstupy.

Dřívější pokusy



Abychom se mohli poučit z předchozího úsíli, pojďme si shrnout dřívější pokusy. Díky tomu, že do OSM je možné používat i neschválené tagování, mohli autoři vždy i vytvořit mnoho příkladů použití.

Level relation – 5/2009



Kratký návrh na vytvoření relace sdružující všechny objekty na jednom patře, s tím že více patrové objekty by byly ve více relacích. Tento koncept se opakuje i v následujících proposals. Základ tagu level=* s významem level=0 jako přízemí a možností tagování mezipater desetinným číslem.

K dispozici na: wiki.osm.org/Relations/Proposed/Level 

Indoor by saerdnaer – 3/2011



Komplexnější návrh z Technické univerzity v Mnichově opět využívá relace pro jednotlivá patra, ale zavádí některé další užitečné koncepty.



Obrázek: desetiný level u schodů; přehled relací pater

Jednotlivé místnosti označuje plochami s tagem room=*, kde se využívá rozlišení typů dle SIG3D (součást CityGML – viz kapitola 1.1.8). Dále využívá již dobře ustanovené tagy: ref=* pro číslo místnosti, entrance=* pro vchody a access=* pro omezení vstupu.

Chodby nemodeluje jako plochy, ale liniově vedené cesty s novým tagem highway=corridor. Tak je to v OSM zvykem a zároveň je tím tvořen routovací graf. Pro venkovní vedení přidává outdoor=yes.

Pro level=* ukazuje jak by mohlo být tagováno schodiště, zejména v případě, že na různých odpočívadlech vedou dveře jinam. Kromě uvádění levelu v relacích, také uvádí alespoň jeden level=* u samotného prvku.

K dipozici na: wiki.osm.org/Proposed\_features/indoor pouze v německém jazyce

Naše hodnocení: Dobře udělaný proposal, který už je nahrazen jinými. Využíval princpů OSM (uzly, cesty) a existujících tagů (access,…), měl přesah do CityGML a využíval pokročilejších vlastností tagu level, který i přenášel na jednolivé prvky.

Level\_map – 9/2011



Zcela vyčerpávající proposal čítající 8 stran textu popisuje poměrně složitý systém využití relací a rolí. Využívá též možnosti replikace objektů, které se nachází ve více patrech na stejném místě (např. chodba, toalety). Objekt je pak v OSM databázi jen jednou, ale je přiřazen do více pater.

Hlavním kamenem je relace type=level\_map. Přiřazuje uspořádaný seznam id pater v objektu do tagu levels=*, např. levels=B;G;1-3 (pro Basement, Ground floor a patra). Zde nabízí i možnost rozšířeného pojmenování pater pomocí levels=B=Basement;… či specifikaci výšky v metrech a dalších jazykových překladů.

Všechny objekty v budově pak jsou členy této relace a jejich role udává patro. Proposal nabízí i speciální role pro spojení více pater (schodiště), či dokonce možnost definovat nové pojmenované role s pokročilými možnostmi. Do rolí též navrhuje psát možné odlišnosti v tagování objektu v různých patrech, např. role:Toilets=B;3[female=yes;male=no];4[female=no;male=yes].

K dipozici na: wiki.osm.org/Relations/Proposed/Level\_Map

Naše hodnocení: návrh není vhodný, kvůli své složitosti. Vytváří zbytečně několik nových syntaxí uvnitř (textového) tagu, zavádí identifikátory pro složitější popis rolí a nevyužívá tedy dobře existující architekturu prvků a tagů. Implementace editoru i prohlížeče by byla velmi náročná.

IndoorOSM – 11/2011



Jeden z úspěšnějších proposals z Universität Heidelberg[46] dal za vznik i editoru a prohlížeči. V současné době je zamítnutý ve prospěch Simple Indoor Tagging.

Definuje jednu hlavní relaci pro budovu type=building, která sdružuje podrelace pater a zároveň obsahuje další informace o budově, včetně některých navržených vlastností pro 3D vzhled. Samotná relace patra type=level pak udává jméno a číslo patra a sdružuje jednotlivé objekty.



Obrázek: Relace budovy a její podrelace

Všechny indoor objekty (pokoje, chodby, atd.) navrhuje tagovat buildingpart=*, např. buildingpart=room apod. a též je umístit do relace patra v roli buildingpart. Dveře a okna jsou zvoleny jako uzly s možností vyplnit šířku, výšku, omezení vstupu apod. Prvky mají definováno patro pouze přes příslušnost k relaci, tag level=* na nich tedy není vyžadován.

Propojení mezi patry je řešeno netradičním způsobem. Každý prvek, který se propojuje do jiného patra má na sobě tag s ID uzlu “kam je možné jít”, např. connector:ids=67890. Tento “connector” je tedy jednosměrná informace pro routing. Pokud by šlo o schodiště, tedy obousměrný segment, bylo by nutné vzájemně referencovat vždy uzly v patrech nad sebou.

Díky větší aktivitě týmu z univerzity i dalších zapojených lidí bylo otagováno velké množství budov nejen v Německu (dle taginfo existuje téměř 6000 relací type=building[47]) a vznikla řada nástrojů. Podívejme se na některé z nich:

Javascriptový prohlížeč postavený na Overpass API: http://clement-lagrange.github.io/osmtools-indoor/ 

Indoor 3D – prohlížeč indoor dat na systému OSMBuildings: osmbuildings.org/indoor

Univerzitní 2D i 3D prohlížeč: http://indoorosm.uni-hd.de/ a http://indoorosm.uni-hd.de/3d/ 

Simulátor evakuace budovy v prostředí MATSim

Návod na editaci v editoru JOSM na stránce proposalu.

K dipozici na: wiki.osm.org/IndoorOSM

Naše hodnocení: Tento proposal komplexně řeší problematiku indoor mapování, uvažuje nad možností vykreslování ve 3D i navigací. Zadávání pomocí relací není snadné, ale je možné, jak se i prokázalo praxí. Hlavní nevýhodou je referencování ID uzlů, to autoři později sami označili za nevhodné, neboť toto ID se může měnit. Jistá nevýhoda též může plynout ze zaměnitelnosti tagu buildingpart s building:part, který je používaný pro metodiku Simple 3D buildings.[48]

Compound facility – 11/2013



Tento návrh z Budapest University of Technology and Economics vznikl zejména pro potřebu navigace cestujících v rámci rozsáhlých stanic. Univerzitní tým připravoval multimodální plánovač tras a tento proposal měl nastavit metodiku, jak tagovat indoor mapy s ohledem na složitější modely pater a hledání tras v napojených plochách.

Navrhuje podobný systém relací jako IndoorOSM. Hlavní relace (type=compound\_facility) sdružuje všechny vchody a podrelace jednotlivých pater (type=level). Patra nejsou primárně organizovány tagem level=*, ale novým tagem zlevel=*, ten má číslovat jednotlivé z-roviny, protože na nádražích často patra nejsou dobře rozlišitelná.

Nově přidává relaci type=convex\_area, která v konvexních plochách umožňuje zadávat refereneční body pro routing (jinak je hledání vhodné pěší trase v plošném objektu chodby poměrně náročné).

Spojení mezi patry modeluje další relací type=level\_connector se dvěma rolemi: from a to.

Součástí návrhu je i ukázka mapování v JOSM a dostupná ukázková data pro jedno nádraží. Ovšem data nikdy nebyla vložena do OSM a zřejmě posloužila jen pro teoretické účely.

K dipozici na: wiki.osm.org/Proposed\_features/CompoundFacility

Naše hodnocení: Návrh opět (nad)užívá relací, není vhodný pro snadné mapování. Místo zlevel=* lze výhodněji použít desetinná místa pro tag level=*. Zajímavá je myšlenka s referenčními body v složitých plochách, ovšem i to lze řešit “chytře” čistě algoritmicky.

Simple Indoor Tagging – 6/2014



Nejnovější návrh z komunity je v lecčems převratný. Upustil od používání relací a místo toho pouze striktně všude doplňuje tag level=*. Správně podotýká, že typický vývoj mapy v OSM začíná na “přibližném odhadu” a končí “šílenou detailností”. Též je dobře kompatibilní s metodikou Simple 3D.



Obrázek: Jednoduché tagování bodů zájmu; detailní tagování místností

Pro jednoduché tagování bodů zájmu tedy navrhuje pouze přidat na obrys budovy min\_level=* a max\_level=*, které určí všechna patra budovy. Jednotlivé uzly uvnitř pak mají pouze nastavený level=*.  Patra by měla korespondovat s označením patra v budově. Typicky level=0 znamená přízemí, ale nemusí tomu tak být. Pokud v budově některé patro chybí, může to být označeno tagem non\_existent\_levels=*.  Zcela odebírá nutnost používání relací, ale nezabraňuje jejich využití pro seskupení budovy a patra.

Navrhuje komplexnější mapování vnitřních prostor pomocí tagu indoor=*, například pokoj indoor=room má implicitně stěny, zatímco chodba indoor=corridor či jiná plocha indoor=area stěny nemá. Samostatně lze též stěny doplnit pomocí indoor=wall. Tak je možné pokrýt celé území patra a modelovat vše dle potřeby. K označení čísla místnosti slouží tag ref=*, název klasicky name=*.

Spojení mezi patry je navrženo plochou, která se rozkládá přes více pater (např level=0-3). Každé patro je pak napojeno průchodem (door=yes + level=0). Routování je tedy umožněno přes všechny dotyky ploch beze stěn a též všemi průchody.

Dodatečné informace k patru je možné zadávat k obrysu patra s tagem indoor=level, tam lze snadno propojit číslo patra level=* s jeho pojmenováním name=*, výškou height=* a potažmo dalšími 3D vlastnostmi.

Pro výškové budovy se často rozložení chodeb, dveří a dalších prvků opakuje. I na to autoři mysleli každý indoor prvek, který má definováno patro, může mít i tag repeat\_on=*, který značí patra kde je jeho úplná kopie. (Zde došlo v průběhu psaní práce ke změně – tag level=* musí být přítomen vždy.)

Metodika se díky jednoduchosti ujala, ale ještě není považována za finální a tedy ani nezačal proces hlasování. V první půlce roku 2015 vytvořil uživatel Panier Avide prohlížeč postavený na Overpass API – openlevelup.net. Indoor editor vznikl v době dokončování této práce.

K dipozici na: wiki.osm.org/Simple\_Indoor\_Tagging

Naše hodnocení: Tento proposal je vhodný k následování, neboť vnímá filozofii OSM s možností postupného zlepšování zmapovanosti. Nejlépe ze všech využívá existující architekturu a nevytváří komplikované relace. Místo striktního “flexibilního” oddělení pater relacemi využívá faktu, že patra lze modelovat jako přirozená čísla a jsou tedy předem známé[an].

Spojení pater přirozeně modeluje například výtahovou šachtu – ovšem pro schodiště chybí možnosti informace o vedení schodů.

Poznámka k tagu level=*



V závěru roku 2011 se s poněkud nejistým procesem objevila stránka o tagu level=*, nejdříve pouze odkazovala na proposals, které ho využívaly. Ovšem při postupném doplňování se z ní stala “uznaná vlastnost”. Nejspíše je to dáno její jednoduchostí a tím, že tag je intuitivně používaný. V současnosti má téměř 300 000 výskytů.

Naše metodika pro indoor mapy[ao]



Naše řešení v principu velmi vychází z Simple Indoor Tagging, ale chceme navrhnout jakési jádro indoor tagování, které nechá větší volnost ve volbě indoor prvků. Zejména doplňujeme robustně model level=* a repeat\_on=* a adresujeme některé další potíže.

Budovy

U budov (building=yes) i částí budov (building:part=yes) se zachovává min\_level=* a max\_level=*. Též přípapadné non\_existent\_levels=*.

    

Obrázek: využití tagu level a repeat\_on; mezipatra pomocí desetinného čísla

Patra

Nestandardní situací v OSM je, že formát level=* musí být srozumitelný pro lidi, aby šel intuitivně zadávat, ale musí být samostatně srozumitelný i pro počítače, aby šlo patra seřadit a tak vykreslit. Souhlasíme tedy se Simple Indoor, který navrhuje označování pater odpovídajícím číslem, např. v budově s patry S, P, 1, 2, bude patro “S” odpovídat level=-1 a patro “P” level=0.

Označení pater je umožněno i v desetiném formátu. Viz obrázek. To je vhodné pro tzv. mezaninové prostory a též třeba pro lepší vedení schodů. (Obrázek v kapitole “Indoor by saerdnaer”)

Existence patra je implicitně předpokládána, kdykoliv v něm je libovolný prvek. Není tedy nutné vždy vytvářet budovu ani patro. Naopak v případě potřeby zaznačení více informací k patru, využíváme obrys indoor=level, s tagy ref=*, name=*, height=* atd.

Každý prvek pro indoor musí mít tag level=*. V syntaxi umožňujeme rozsah (0-2) pro rozpětí prvku přes více pater, např. výtahová šachta. Takový prvek se pak nachází v celém intervalu, tedy i v případných desetinných mezipatrech. Dříve využívaný výčet (0;1;2) se nedoporučuje, ale z důvodu zpětné kompatibility ho interpretujeme jako diskrétní hodnoty.

Navíc každý prvek může mít tag repeat\_on=*, ten značí jeho přesné opakování v jiných patrech včetně nastaveného rozpětí. (Viz obrázek.) Syntaxe je doporučená výčtem (0;1;2). Využití rozsahu (0-2) je možné, ale znamená opakování ve všech patrech od minima s “násobky jedničky”. Tedy repeat\_on=0.5-3  je ekvivalentní repeat\_on=0.5;1.5;2.5.

Rozsahy jsou pro oba tagy možné i pro záporné hodnoty, např. level=-1-5, či pro obě záporá čísla level=-4--1, vždy ovšem tak, že první hodnota je menší než druhá.

Přízemní patro není vyžadováno – číslování je dáno číselným označením v budově. Pro routing má pak význam informace v kterém patře je uzel entrance=main napojený na venkovní cesty.

Místnosti a chodby

Chodby je nově možné tagovat jako klasickou cestu highway=footway. Považujeme totiž za zásadní dodržovat filozofii OSM – mapovat od jednoduchého k přesnějšímu. V tomto je Simple Indoor příliš důkladný a rovnou nutí tvořit plošné vnitřní prostory.

Kromě chodeb často existují jakési nádvoří či atria. Ty je vhodné otagovat plošnou komunikací jak je v OSM zvykem highway=footway + area=yes.

Specifikem budov je, že všechno je průchozí. Navrhujeme tedy oddělit primárně průchozí oblasti (chodby), které budou značeny standardní komunikací, a primárně neprůchozí (místnosti), které budou značeny indoor=* prvky dle Simple Indoor.

Místnosti je možné značit jako uzly nebo jako plochy  pomocí indoor=*, či třeba pouze dveře door=yes. Opět dle filozofie od jednoduchého k přesnějšímu. Dle možností je vhodno doplnit tag ref=* či name=*.

Schody navrhujeme značit standardně highway=steps, typicky s rozsahem přes více pater level=0-1. Aby se určilo, kde je začátek a konec, navrhujeme využít orientace cesty, s tím, že směr povede od nižšího patra k vyššímu – tedy šipka míří dokopce. Totéž platí pro libovolné cesty, které jsou vedeny přes více pater.

Shrnutí

Díky zjednodušení je možné využívat všechny prvky běžné v OSM světě pouze s přidáním tagu level=*. Obrysy budovy jsou vhodné pro zobrazení ve webových mapách a umožňují i navigaci a geocoding v budově. Zachována je i kompatibilita s metodikou Simple 3D.

Pro konzumenta dat stačí v OSM hledat všechny prvky s tagem level=* a min/max\_level=*, následně už je možné vytvořit uživatelsky přívětivý přepínač pater a zobrazovat jen relevantní data. Zde podotýkáme, že indoor prohlížeče z principu musí fungovat na vektorových datech, neboť vyžadují vykreslování v prohlížeči. Adepti na vektorové prohlížeče jsou technologie Mapbox GL, OSMBuildings JSON či mobilní aplikace typu Maps.me a Osmand.

Klasické 2D mapy jsou většinou předgenerované obrázkové dlaždice. Díky využívání tagů běžných komunikací jsou ovšem tyto vykresleny a tak je většinou zachována informační hodnota i pro ně.

Editory pro toto schéma zatím neexistují a využívání rozsahů v tagu level=* nám brání snadno editovat v editoru JOSM (narozdíl od relací). Je tedy vhodné upravit stávající editory, aby schéma podporovaly, úpravě iD se věnujeme v další kapitole.



Obrázek: zobrazení otagovaného patra 13 v budově ČVUT (editor JOSM)

2.5. (3D)



// jen kdyby bylo málo stránek a moc času

2.6. (Komerční poskytovatelé OSM related služeb)



// mapbox (spousta aktivit… koordinace iD), mapzen, systeme-d, mapquest, rendery, atd

3. Úprava editoru iD



V předchozí kapitole jsme položili důkladný úvod do architektury OSM, shrnuli předchozí proposals a navrhli novou metodiku pro indoor mapy. Tato kapitola pojednává o implementaci metodiky do editoru iD. Bylo třeba nastudovat pokročilou knihovnu d3.js, velmi sporou dokumentaci iD-core a pochopit samotné fungování z kódu a debugingu. Po návrhu uživatelského rozhraní mohla začít implementace v podobě přidání ovládacích prvků pro indoor mód a úpravy vykreslovacího enginu. Nakonec bylo možné sepsat pull-request do GitHubu tohoto projektu a zasadit se o jeho začlenění.

3.1. Popis editoru a technologií



O editoru jsme se již rozepsali[ap] v kapitole 2.2. Shrňme nyní, že se jedná o webový editor postavený na několika technologiích z rodiny HTML5. Cílí na začátečnickou skupinu uživatelů a proto má propracované jednoduché uživatelské rozhraní. Integrační a unit testy jsou postavené na frameworku Mocha.

Pro úpravu bude též zásadní vlastnost filtrování zobrazených prvků dle kategorií, tzv. Features.

Vizualizační knihovna d3.js 



Knihovna Data-Driven Documents (d3js.org) se zabývá navázáním datových polí (arrays) na objektový model HTML dokumentu (DOM). Pro přídání a odebrání prvků do pole definuje enter \& exit transitions, které deklarativně určují co se děje při dané operaci. Dokumentace knihovny je důkladná a díky své oblíbenosti lze najít i mnoho dalších zdrojů, přesto však bylo těžké do nezvyklého konceptu proniknout a začít v něm programovat[aq].

API knihovny je podobné známému jQuery, bez navázání dat můžeme říci, že totožné. Zde selektor a mutační funkce:

d3.selectAll("p").text("Hello");

Síla knihovny se ukáže při navázání dat, následně totiž můžeme do mutačních funkcí místo hodnot uvádět funkce. Při změně dat na stejném selektoru se pak automaticky provedou operace za enter() funkcí, zde přidání odstavce a nastavení jeho textu dle datové hodnoty.

d3.select("body").selectAll("p")

   .data([4, 8, 15, 16, 23, 42])

   .enter().append("p")

    .text(function(d) { return "I’m number " + d + "!"; });

Základ je tedy jednoduchý, ale pro zápis animací a složitějších vnořených struktur vyžaduje jistý cvik. Abstrakce objektového modelu jako data je velmi užitečná a hodí se nejen pro vykreslení dat v mapě, ale i některých ovládacích prvků.

Architektura iD[49][ar]



Autoři zvolili na začátku vykreslování dat pomocí inline SVG – zápis obrázku pomocí XML, které je možné přímo inline vložit do HTML DOM. Velkou výhodou pak je, že SVG DOM generuje události jako HTML objekty, lze ho stylovat pomocí klasických selektorů CSS3 a lze ho pochopitelně vytvářet stejně jako HTML DOM. Přesně tyto výhody daly i za vznik knihovně d3.js pro abstrakci vizualizací nejen nad SVG.

Druhá možnost by byla používat tzv. HTML canvas, jehož vykreslení je mnohokrát rychlejší, nenáročné na paměť a podporované i staršími prohlížeči. Ovšem v tu chvíli by autoři museli programovat všechny věci ohledně grafického enginu sami. S SVG naopak využili grafické jádro prohlížeče. Volba padla zejména pro větší flexibilitu a snadnější vývoj.

Kromě samotné knihovny d3.js využívá iD i několika rozšíření: d3.xhr() pro kladení AJAX požadavků, d3.dispatch() pro volání callback funkcí dle návrhového vzoru observer, d3.geo.path() pro generování SVG linek dle souřadnicí a zadané projekce a d3.behavior.zoom() pro zoomování a posouvání mapy pomocí CSS3 transformací.

Core architekura



Jádro editoru se týká zejména datového modelu postaveného nad OSM XML API. Využívá několik velmi hezkých vlastností a je velmi důkladně otestován.

Entity odpovídají základním prvkům OSM: iD.Node, iD.Way a iD.Relation. Jako jediné v projektu využívají dědičnosti a to od iD.Entity. Lze tak přistupovat k vlastnostem id, tags a dalším dle typu. Díky odlišnému namespace pro id OSM typů zavádí jednotný namespace identifikátorů, kde prefixuje číselné id prvním písmenem typu, např. n123, w42, r30.

Velmi chytře navrženou částí je implementace historie. Podobá se verzovacímu systému GIT. Každá entita je po vytvoření imutabilní, to znamená, že tu konkrétní instanci objektu už nelze změnit. Entity jsou sdruženy v imutabliním grafu, který ukazuje na jednotlivé imutabilní objekty. Při “úpravě” jedné entity je tak vlastně vytvořena nová entita s upravenými vlastnostmi a vytvořen nový graf, který ukazuje na všechny původní entity a jednu novou.

Graf tedy odpovídá GIT tree a entity jednotlivým GIT objects. Tato implementace je velmi šikovná na funkci Undo, nabízí dobrou abstrakci a též je úsporná na operační paměť.



Obrázek: Historie a graf jako imutabilní objekt

Datový model nabízí několik dalších šikovných funkcí:

iD.Difference(g1,g2) – vrací změněné prvky mezi dvěma grafy, což je potřebné pro vykreslení změn

iD.Tree(g) – obalí daný graf datovou strukturou R-tree a následně umí efektivně vrátit entity ve zvolném souřadnicovém rámečku BBOX. Složitost vyhledání O(logMn), maximální počet listů byl zvolen M=9. 

příklad: var features = tree.intersects(iD.geo.Extent([0, 0], [2, 2]), tree.graph());

iD.Action() – abstrakce nad více opracemi na grafu, vytváří nový graf, který uloží do historie. Např. při smazání cesty se odeberou i všechny její uzly. iD.Operation() pak abstrahuje akci na operaci proveditelnou tlačítkem či klávesovou zkratkou a hlídá jestli je možné v danou chvíli akci provést.

iD.Mode(), iD.Behaviour() – základní mód je prohlížení mapy a tlačítky je možné zvolit tři módy pro přidávání uzlů, cest a ploch. Módy definují chování jako například zvýraznění po najetí myší na prvek, apod. (Poznámka: vyvíjený Indoor mód není módem v tomto “core” smyslu)

samotné vykresování SVG zajišťuje iD.Map, která implementuje příslušné: iD.svg.Points, iD.svg.Vertices, iD.svg.Lines a iD.svg.Areas.

3.2. Návrh uživatelského rozhraní



První iterace



Uživatelské rozhraní bylo třeba vyvíjet v několika iteracích a výsledky testovat. První návrh proběhl na papíře a zde je prezentován pomocí softwaru Blasamiq Mockups.



Obrázek: První návrh uživatelského rozhraní včetně popisu interakcí

Následně proběhlo několik diskusí a testů a rozhodli jsme se návrh přehodnotit.

Druhá iterace



Pro uživatele by bylo nešikovná nutnost vybrat budovu, ačkoliv to je implementačně snazší (přepínač pater nabízí pouze patra pro konkrétní budovu). Navíc v naší metodice umožňujeme i tvorbu level=* prvků bez ohraničující budovy, což by se muselo adaptovat.

Tlačítko se tedy bude jmenovat “Indoor mód” a bude se ukazovat vždy při vybrané budově či prvku s level=*. Indoor mód se zapne přesně do patra vybraného prvku.

Přepínač pater bude aktivní napříč celou aplikací – to také intuitivně odpovídá umístění v záhlaví, nikoliv u zvoleného prvku. Nabízet se tedy budou všechna dostupná patra ve výřezu.

V indoor módu se nyní schovají všechny nesouvisející prvky, aby uživatel věděl čeho se týká jeho editace.

Vyznačení plochy budovy bylo rozšířeno i vně, aby ji šlo vybrat při plném pokrytí indoor plochami.



Obrázek: Upravený editor po druhé iteraci a zobrazené operace





Obrázek: Finální editor po čtvrté iteraci v patře 0 a v patře -1

Třetí a čtvrtá iterace



Pro jednoduchost shrnujeme výsledky a obrázek po obou iteracích a tedy náš závěrečný editor. Více obrázků je k dispozici v další kapitole.

Pro uživatele bylo matoucí skrytí všech okolních prvků, tedy ponechány[as] všechny komunikace.

Aktivní ortofoto vrstvu jsme pouze utlumili, stejně tak všechny komunikace bez level=*. Při přechodu do záporných pater pak obojí utlumeno podstatně více.

Pro lepší vnímání výšky zobrazujeme jen ty budovy, které mají příslušná patra (building:levels=*). Budovy s min/max\_level=* jsou navíc zvýrazněné, neboť do nich je vhodné tagovat dle indoor metodiky.

Přepínači pater jsme přidali tlačítka pro zvýšení a snížení patra o jedničku. Kvůli častému používání jsme vytvořili větší tlačítka vedle sebe. Pro editor je tento přístup přirozenější než výběr z existující pater, protože přirozeně vede uživatele k tomu, aby v novém patře začal mapovat.

Plochy indoor=corridor a indoor=area, tedy ty beze stěn, zobrazeny slabším ohraničením.

Zvolené patro přidáno do URL pro možnost odkazovat na zobrazení konkrétního patra v indoor módu.

Čtvrtá iterace se týkala pouze drobných úprav přepínače:

Tlačítka pro zvýšení a snížení patra jdou přes desetinná patra, v případě, že taková existují.

 V nabídce comboboxu se zobrazují pouze ty patra, která jsou explicitně definovaná v tagu level=*. Pomáhá to uživateli ve zjištění, která patra jsou zmapovaná.

3.3. Implementace



Bylo potřeba přidat ovládací prvek pro aktivaci Indoor módu a náslenou volbu patra. Dále pak zajistit skrytí prvků v jiných patrech.

Nejdříve se zdálo vhodné upravit třídu iD.map v místě, kde posílá všechny entity k filtrování Features a dál k  vykreslení SVG:

if (context.indoorMode()) {

  data = data.filter(function (entity) {

      return inRange(context.indoorLevel(), entity.tags.level) || … ;

  });

}

data = features.filter(data, graph);   // filtrování dle Features

svgSurface.call(drawVertices, graph, data, filter, map.extent(), map.zoom())...

Ovšem záhy se toto řešení ukázalo jako nevhodné. Toto volání se provádí velmi často – při každé události context.enter (vybrání prvku, úprava atd.), při map.move (posun mapy tzv. debounced na 400 ms) a dalších.  Vykreslení pomocí d3js je rychlé (vykreslí se jen DOM pro změny), Features využívá kešování. Ovšem náš přístup by při každém volání musel naparsovat level=* tag pro všechny prvky a zjistit, zda se nachází v rozsahu.

Bylo tedy třeba hledat jiné řešení. To jsme našli v podobě již fungujícího filtrování Features. Logika je poměrně složitá, protože kromě skrytí prvků dle zašrtlých kategorií se zde provádí i tzv. autohide. Ten automaticky skrývá prvky když by jich na obrazovce bylo příliš mnoho na čtvereční jednotku. Navíc jeden prvek může patřit i do více kategorií a je skryt pokud je skyrta alespoň jedna z nich.

Definovali jsme tedy kategorii indoor a indoor\_different\_level, kterou bylo třeba při zapnutí indoor módu automaticky schovat:

defineFeature('indoor', function isIndoorOther(entity) {

  return !!entity.tags.level;

});

defineFeature('indoor\_different\_level', function isHiddenByLevel(entity, resolver, geometry) {

    return inRange(context.indoorLevel(), entity.tags.level) || … ;

});

Tím jsme využili mnoho stávající logiky a zároveň při vstupu do Indoor módu přirozeně i mohli vypnout nežádoucí Features.

Následně bylo třeba správné vykreslení uzlů s vlastnostmi, skrývání uzlů v cestě dle jejích vlastností, vynucení skrytí i pro nově přidané prvky a mnoho další vývojové i ladící práce, která je dostupná ve zdrojových kódech.

Práci mimo jádro (tedy vykreslovací vzhledy, presets a další) zde nerozepisujeme, neboť není tak technicky náročná – popis se nachází v předchozí kapitole a samozřejmě je i k nahlédnutí diff zdrojových kódů.

3.4. Začlenění do projektu iD



Naše úpravy proběhly na forku GIT repozitáře. Po dokončení třetí iterace byl zaslán tzv. pull-request. Jedná se o standardní formu jak na GitHubu nabídnout správci nějakého projektu začlenění nového kódu. Před posláním pull-requestu byl proveden git rebase na aktuální master, aby se předešlo konfliktům kódu.

Pull request je k dispozici na adrese https://github.com/openstreetmap/iD/pull/3097 a otisk ke stavu odevzdání v příloze této práce a na CD.

Hlavní správce projektu Bryan Housel na něj reagoval, že se mu práce líbí a rád by jí do projektu v nějaké formě začlenil. Ovšem nejdříve bude třeba vymyslet jak lépe pojmout uživatelské rozhraní – rád by aby tlačítko bylo dostupné stále, ale trochu více schované.

Na dalším vývoji budeme spolupracovat i nad rámec této diplomové práce.



Obrázek: První komentář správce projektu





4. Využití výsledků práce  (ještě dopisuju)



4.1. Renderery, tisk



4.2. 3D viewer





Obrázek: Koncept 3D prohlížeče – upraveno v grafickém editoru

4.3. ČVUT



4.4. Úřady a muzea



//např. Národní muzeum dříve projevilo zájem o ČVUT Navigátor, konkr. právě tu lokaci QR kódy





Závěr



...

Vyšli jsme tedy maximálně ze stávajícího systému Simple Indoor a navrhli některá zjednodušení pro vhodnější zadávání uživateli. Systém byl ozkoušen nejprve ve stávajícím editoru JOSM a poté namodelováno několik případů v námi upraveném editoru iD.

Přílohy:


